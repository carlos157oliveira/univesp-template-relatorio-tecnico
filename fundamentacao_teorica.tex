Pesquisar em fontes confiáveis como monografias, trabalhos de conclusão de
cursos, artigos científicos, revistas especializadas, dissertações e teses e entre outras fontes
como instituições públicas ligadas às normatizações.

A fundamentação deve ser condizente com o problema em estudo.

Busque e cite fundamentos relevantes e atuais sobre o assunto a ser estudado e
demonstra o entendimento da literatura existente sobre o tema.

As citações devem estar acompanhadas com as especificações da fonte (AUTOR,
ano, página). As citações e paráfrases devem ser feitas de acordo com as regras da ABNT
6023, de 2002; para as citações literais com mais de três linhas, devem ser utilizados fonte
nº 10, com recuo de parágrafo 4 cm (conforme exemplo adiante).

\begin{citacao}

Faz necessária a busca por alternativas para dinamizar o processo de ensino-
aprendizagem em que o professor e os alunos sejam sujeitos e caminhem juntos na aventura de aprender e descobrir o novo e vejam sentido nos seus fazeres e não simplesmente no cumprimento de mais uma tarefa. A matemática, portanto, faz
parte da vida e pode ser aprendida de uma maneira dinâmica, desafiante e
divertida. (PILETTI, 1998, p. 102)

\end{citacao}
