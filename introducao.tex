A Introdução é a apresentação do assunto a ser tratado, também deve conter também
o Problema a ser pesquisado.

Ao desenvolver a introdução, o grupo deve explicar o assunto que deseja:

\begin{itemize}

\item Desenvolver o tema
\item Anunciar a ideia básica
\item Delimitar o foco da pesquisa
\item Situar o tema dentro do contexto geral da sua área de trabalho
\item Descrever as motivações que levaram à escolha do tema
\item Indicar o objeto do trabalho: O que será estudado?

\end{itemize}

O Texto do trabalho deve conter a formatação indicada neste documento:

\begin{itemize}

\item FONTE, TAMANHO E COR: Times New Roman, tamanho 12 para texto, 10 para
citações de mais de três linhas e de 10 para notas de rodapé; \textleftarrow Cor preta
\item MARGENS – superior e esquerda de 3cm; inferior e direita de 2cm.
\item TÍTULOS OU SUBTÍTULOS – alinhados à esquerda, iniciando sempre uma nova
página.
\item PAGINAÇÃO (números das páginas) – Inferior à direita começando da introdução
em algarismos arábicos (1, 2, 3....). Todas as letras dos títulos dos capítulos devem
ser escritas no canto esquerdo de cada página, em negrito e maiúsculas.
\item ESPAÇAMENTO \textleftarrow Todo texto deve ser digitado em espaço 1,5. \textleftarrow Excetuam-se:
citações longas (com mais de três linhas) notas de rodapé, as Referências
Bibliográficas (ou Bibliografia) e as legendas de ilustrações e tabelas, que são
digitadas em espaços simples. Os parágrafos devem ser separados por uma linha em
branco. Citações com mais de três linhas, fonte tamanho 10, espaçamento simples e
recuo de 4cm da margem esquerda. Notas de rodapé, fonte tamanho 10.

\end{itemize}