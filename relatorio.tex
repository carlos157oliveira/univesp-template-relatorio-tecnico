% texlive, texlive-publishers, texlive-lang-portuguese

\documentclass[a4paper,openright, oneside, article, 12pt, brazil]{abntex2}

\usepackage{indentfirst}
\usepackage[alf, abnt-full-initials=yes]{abntex2cite}

\usepackage{amsmath}
\usepackage{graphicx}
\usepackage{float}
\usepackage{univesp_preambulo}
\usepackage{fancyhdr}
\usepackage{subcaption}


\titulo{Título do trabalho (Frase curta que sintetize o foco do trabalho)}
\tituloabstract{Work title (short phrase that summarizes work focus)}
\curso{(incluir seu curso)}
\autor{
    Fulano Fulanado \\
    Siclano Siclanado \\
    Beltrano Beltranado
}
\autoresbibliografia{
    FULANADO, Fulano;
    SICLANADO, Siclano;
    BELTRANADO, Beltrano;
}
\local{Cidade - SP}
\data{ano}
\tutor{(Nome do Tutor)}

% O link do youtube pode ser omitido (apague a linha ou comente-a)
% Quando usado, desenha uma caixa e coloca o link dentro conforme padrão
\linkyoutube{https://www.youtube.com/video-qualquer}

\begin{document}

\imprimircapa

\imprimirfolhaderosto

\imprimirresumo
{Até 250 palavras incluindo: breve introdução, objetivos, metodologia adotada, resultados obtidos e considerações finais. Formatação: Espaçamento simples, parágrafo único.
}
{Palavra 1; Palavra 2; Palavra 3; Palavra 4}

\imprimirabstract
{Maximum 250 words including: short introduction, objectives, adopted methodology, results obtained and final considerations. Format: Simple spacing, unique paragraph.
}
{Keyword 1; Keyword 2; Keyword 3; Keyword 4}
	
\newpage
\listoffigures*

\newpage
\tableofcontents*

\textual

\pagestyle{fancy}
\fancyhf{}
\fancyhead[R]{\thepage}
\fancyhead[L]{\nouppercase{\rightmark}}

\newpage
\section{INTRODUÇÃO}
A Introdução é a apresentação do assunto a ser tratado, também deve conter também
o Problema a ser pesquisado.

Ao desenvolver a introdução, o grupo deve explicar o assunto que deseja:

\begin{itemize}

\item Desenvolver o tema
\item Anunciar a ideia básica
\item Delimitar o foco da pesquisa
\item Situar o tema dentro do contexto geral da sua área de trabalho
\item Descrever as motivações que levaram à escolha do tema
\item Indicar o objeto do trabalho: O que será estudado?

\end{itemize}

O Texto do trabalho deve conter a formatação indicada neste documento:

\begin{itemize}

\item FONTE, TAMANHO E COR: Times New Roman, tamanho 12 para texto, 10 para
citações de mais de três linhas e de 10 para notas de rodapé; \textleftarrow Cor preta
\item MARGENS – superior e esquerda de 3cm; inferior e direita de 2cm.
\item TÍTULOS OU SUBTÍTULOS – alinhados à esquerda, iniciando sempre uma nova
página.
\item PAGINAÇÃO (números das páginas) – Inferior à direita começando da introdução
em algarismos arábicos (1, 2, 3....). Todas as letras dos títulos dos capítulos devem
ser escritas no canto esquerdo de cada página, em negrito e maiúsculas.
\item ESPAÇAMENTO \textleftarrow Todo texto deve ser digitado em espaço 1,5. \textleftarrow Excetuam-se:
citações longas (com mais de três linhas) notas de rodapé, as Referências
Bibliográficas (ou Bibliografia) e as legendas de ilustrações e tabelas, que são
digitadas em espaços simples. Os parágrafos devem ser separados por uma linha em
branco. Citações com mais de três linhas, fonte tamanho 10, espaçamento simples e
recuo de 4cm da margem esquerda. Notas de rodapé, fonte tamanho 10.

\end{itemize}

\newpage
\section{DESENVOLVIMENTO}

\subsection{PROBLEMA E OBJETIVOS}
O objetivo geral define o que se pretende atingir com o projeto.
Os objetivos específicos definem etapas do trabalho a serem realizadas para que se
alcance o objetivo geral. Os objetivos podem ser: exploratórios, descritivos e explicativos.
Utilize verbos nos infinitivos para os objetivos:

\begin{itemize}
    \item Exploratórios (conhecer, identificar, levantar, descobrir)
    \item Descritivos (caracterizar, descrever, traçar, determinar)
    \item Explicativos (analisar, avaliar, verificar, explicar)
\end{itemize}

\subsection{JUSTIFICATIVA}
Neste item, espera-se que o grupo traga as razões ou práticas que justifiquem a
proposta inicial. Exemplos:

\begin{itemize}
    \item relevância social, cultural e acadêmica;
    \item as contribuições da pesquisa para o local onde o projeto será desenvolvido.
\end{itemize}

\subsection{FUNDAMENTAÇÃO TEÓRICA}
Pesquisar em fontes confiáveis como monografias, trabalhos de conclusão de
cursos, artigos científicos, revistas especializadas, dissertações e teses e entre outras fontes
como instituições públicas ligadas às normatizações.

A fundamentação deve ser condizente com o problema em estudo.

Busque e cite fundamentos relevantes e atuais sobre o assunto a ser estudado e
demonstra o entendimento da literatura existente sobre o tema.

As citações devem estar acompanhadas com as especificações da fonte (AUTOR,
ano, página). As citações e paráfrases devem ser feitas de acordo com as regras da ABNT
6023, de 2002; para as citações literais com mais de três linhas, devem ser utilizados fonte
nº 10, com recuo de parágrafo 4 cm (conforme exemplo adiante).

\begin{citacao}

Faz necessária a busca por alternativas para dinamizar o processo de ensino-
aprendizagem em que o professor e os alunos sejam sujeitos e caminhem juntos na aventura de aprender e descobrir o novo e vejam sentido nos seus fazeres e não simplesmente no cumprimento de mais uma tarefa. A matemática, portanto, faz
parte da vida e pode ser aprendida de uma maneira dinâmica, desafiante e
divertida. (PILETTI, 1998, p. 102)

\end{citacao}


\subsection{APLICAÇÃO DAS DISCIPLINAS ESTUDADAS NO PROJETO INTEGRADOR}
Este item do referencial teórico deve indicar os conteúdos das disciplinas estudadas
no curso que foram abordados no projeto. Espera-se que os estudantes relacionem de forma
clara e coerente, o conteúdo estudado à solução desenvolvido durante o projeto.


\subsection{METODOLOGIA}
Metodologia refere-se aos métodos e instrumentos adotados para a execução do
projeto. Nela, espera-se que o grupo descreva os passos e as estratégias adotadas para o
desenvolvimento do seu projeto integrador.

Assim, indique as estratégias adotadas para:

\begin{itemize}

\item Coleta de dados, indicando em detalhes como ela foi construída: observação,
entrevista, formulário, questionário, etc.
\item Análise dos dados, por exemplo, estratégias referentes à pesquisa qualitativa ou
quantitativa.
\item Selecionar o contexto onde o projeto foi realizado.
\item Elaborar as soluções encontradas para o problema investigado.
Além disso, indique também:
\item O contexto onde o projeto foi realizado;
\item O perfil dos sujeitos participantes, se for o caso;

\end{itemize}

Finalmente, este é o espaço para que o leitor do seu projeto entenda em detalhes,
quais foram as estratégias usadas para que os resultados fossem obtidos.

Importante: quando se tratar de projetos desenvolvidos junto à menores, não são
permitidas a inclusão de fotos dos mesmos, sem a autorização dos pais ou responsáveis.


\newpage
\section{RESULTADOS}

\subsection{SOLUÇÃO INICIAL}
Descrição detalhada com imagens, sobre como se deu o processo de construção da
primeira solução desenvolvida pelo grupo.


\subsection{SOLUÇÃO FINAL}
Descrição detalhada com imagens sobre como se deu o processo de construção da
solução final apresentada pelo grupo. Espera-se que o grupo demonstre quais foram as
melhorias realizadas na solução final, a partir dos feedbacks coletados junto à comunidade
ou local onde o projeto foi desenvolvido.


\newpage
\section{CONSIDERAÇÕES FINAIS}
Deve-se retomar os objetivos e o contexto em que o projeto integrador foi
desenvolvido e apontar os principais resultados obtidos pelo grupo.

Exemplo de citação:
\cite{historia_da_matematica}


\newpage
\postextual

\bibliography{bibliografia}

\anexos

\newpage
\chapter{opcional}
(Materiais coletados por meio de pesquisas em diversas fontes)
Você pode anexar qualquer tipo de material ilustrativo, tais como tabelas, lista de
abreviações, documentos ou parte de documentos, resultados de pesquisas, etc.


\apendices

\newpage
\chapter{opcional}
(Apêndices são criações do autor ou grupo de autores)


\end{document}